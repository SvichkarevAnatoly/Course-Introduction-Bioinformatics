\documentclass{article} % Класс печатного документа

\usepackage{graphicx} % Вставка изображений
\usepackage[utf8]{inputenc} % Кодировка исходного текста - utf8
\usepackage[english,russian]{babel} % Поддержка языка - русского с английским
\usepackage{indentfirst} % Отступ в первом абзаце
\usepackage{tabularx} % Для создания таблицы во всю ширину текста

% Определение нового типа колонки
\newcolumntype{Y}{>{\centering\arraybackslash}X}

\begin{document} % Конец преамбулы, начало текста
\textbf{Студент:} Свичкарев Анатолий Владленович

\textbf{Ген:}
\begin{center}
	\begin{tabularx}{\textwidth}{|c|Y|} \hline
		Названиe & \textbf{P53} (Также известен как: CG10873; CG31325;
			   CG33336; D-p53; Dm-P53; Dmel\textbackslash CG33336;
			   dmp53; Dmp53; DmP53; DMP53; dp53; Dp53; prac) \\ \hline
		Расположение на хромосоме & Chr: 3R, 23049657..23054082,
			   		    комплементарная цепь (Релиз 6.01)\\ \hline
		Функция & белок кодирующая (контроль апоптоза,
			  деление клетки и пролиферация)\\ \hline
	\end{tabularx}
\end{center}

\end{document} % Конец документа
